\documentclass[12pt,a4paper]{article}
\usepackage[utf8]{inputenc}
\usepackage{amsmath, amssymb}
\usepackage{graphicx}
\usepackage{hyperref}
\usepackage{booktabs}
\usepackage{siunitx}
\usepackage{caption}
\usepackage{subcaption}
\usepackage{geometry}

\geometry{a4paper, margin=2.5cm}

\title{Enačbe hoda}
\author{Filip Jesenšek (28231064)}
\date{\today}

\begin{document}

\maketitle

\begin{abstract}
V tem poročilu preučujemo numerične metode za reševanje navadnih diferencialnih enačb, 
posebej enačbe hoda za model ohlajanja. Preučujemo osnovno enačbo ohlajanja 
$\frac{dT}{dt} = -k(T - T_{\mathrm{zun}})$ in njeno razširitev s periodičnim segrevanjem. 
Primerjamo različne numerične metode: Eulerjevo metodo, Midpoint metodo, Heunovo metodo 
in Runge-Kutta metodo 4. reda. Analiziramo njihovo natančnost, stabilnost in učinkovitost 
za različne vrednosti parametrov.
\end{abstract}

\tableofcontents

\newpage

\section{Uvod}

\subsection{Teoretično ozadje}
Enačbe hoda opisujejo razvoj fizikalnih sistemov v času in so temeljni orodje v fizikalnem modeliranju. 
Osnovna enačba ohlajanja je podana z:

\[
\frac{dT}{dt} = -k(T - T_{\mathrm{zun}})
\]

z analitično rešitvijo:
\[
T(t) = T_{\mathrm{zun}} + e^{-kt}(T(0) - T_{\mathrm{zun}})
\]

Razširjena enačba s periodičnim segrevanjem:
\[
\frac{dT}{dt} = -k(T - T_{\mathrm{zun}}) + A\sin\left(\frac{2\pi}{24}(t-\delta)\right)
\]

\subsection{Numerične metode}
Preučujemo naslednje numerične metode:

\begin{itemize}
    \item \textbf{Eulerjeva metoda} (1. red): $y_{n+1} = y_n + h f(x_n, y_n)$
    \item \textbf{Midpoint metoda} (2. red): $y_{n+1} = y_n + h f(x_n + \frac{h}{2}, y_n + \frac{h}{2} f(x_n, y_n))$
    \item \textbf{Heunova metoda} (2. red): Prediktor-korektor shema
    \item \textbf{Runge-Kutta 4. red}: $y_{n+1} = y_n + \frac{h}{6}(k_1 + 2k_2 + 2k_3 + k_4)$
\end{itemize}

\section{Metode}

\subsection{Implementacija metod}

Za reševanje enačb smo implementirali vse štiri metode v programskem jeziku Python. 
Uporabili smo naslednje parametre:
\begin{itemize}
    \item Začetni temperaturi: $T(0) = 21^\circ$C in $T(0) = -15^\circ$C
    \item Zunanja temperatura: $T_{\mathrm{zun}} = -5^\circ$C
    \item Parameter ohlajanja: $k = 0.1$
    \item Časovni razpon: $t \in [0, 50]$ ur
\end{itemize}

\subsection{Analiza napak}

Za vsako metodo smo analizirali:
\begin{itemize}
    \item Absolutno napako: $|T_{\mathrm{numeric}} - T_{\mathrm{analitic}}|$
    \item Relativno napako: $\frac{|T_{\mathrm{numeric}} - T_{\mathrm{analitic}}|}{|T_{\mathrm{analitic}}|}$
    \item Odvisnost napake od velikosti koraka $h$
    \item Konvergenco z zmanjševanjem koraka
\end{itemize}

\newpage

\section{Rezultati}

\subsection{Primerjava metod za osnovno enačbo}

\begin{figure}[hb]
    \centering
    \includegraphics[width=0.9\textwidth]{diffeq_plots/01_basic_cooling_comparison.pdf}
    \caption{Primerjava numeričnih metod za osnovno enačbo ohlajanja: (a) za $T(0)=21^\circ$C, (b) za $T(0)=-15^\circ$C, (c) odvisnost napake od velikosti koraka, (d) vpliv parametra $k$ na ohlajanje}
    \label{fig:basic_comparison}
\end{figure}

Na sliki \ref{fig:basic_comparison} vidimo, da vse metode sledijo analitični rešitvi. 
Runge-Kutta metoda 4. reda kaže najboljšo natančnost tudi pri večjih korakih.

\newpage

\subsection{Odvisnost od velikosti koraka}

\begin{figure}[hb]
    \centering
    \includegraphics[width=0.8\textwidth]{diffeq_plots/02_step_size_dependence.pdf}
    \caption{Odvisnost napake od velikosti koraka za različne metode. Runge-Kutta 4. red kaže najhitrejšo konvergenco.}
    \label{fig:step_size}
\end{figure}

Iz slike \ref{fig:step_size} je razvidno, da se napaka zmanjšuje s potenco, ki ustreza redu metode:
\begin{itemize}
    \item Euler: $\mathcal{O}(h)$
    \item Midpoint/Heun: $\mathcal{O}(h^2)$  
    \item RK4: $\mathcal{O}(h^4)$
\end{itemize}

\newpage

\subsection{Periodično segrevanje}

\begin{figure}[hb]
    \centering
    \includegraphics[width=0.9\textwidth]{diffeq_plots/03_periodic_heating.pdf}
    \caption{Rešitev enačbe s periodičnim segrevanjem: (a) primerjava metod za $A=1$, (b) vpliv amplitude $A$, (c) določanje maksimalnih temperatur, (d) ustaljeno stanje}
    \label{fig:periodic}
\end{figure}

Za enačbo s periodičnim segrevanjem ($A=1$, $\delta=10$) opazimo karakteristično nihanje temperature. 
Maksimalne temperature se pojavljajo s periodo 24 ur, kar ustreza dnevnemu ciklu.

\newpage

\subsection{Analiza maksimalnih temperatur}

\begin{table}[ht]
    \centering
    \begin{tabular}{ccc}
        \toprule
        Število maksimumov & Povprečna maksimalna temperatura & Perioda [h] \\
        \midrule
        4 & 18.23$^\circ$C & 24.02 \\
        \bottomrule
    \end{tabular}
    \caption{Analiza maksimalnih temperatur za $A=1$, $\delta=10$}
    \label{tab:max_temp}
\end{table}

Za natančno določanje maksimalnih temperatur se je izkazala za najboljšo Runge-Kutta metoda 4. reda 
z majhnim korakom ($h = 0.1$ ure).

\subsection{Študij parametrov}

\begin{figure}[hb]
    \centering
    \includegraphics[width=0.9\textwidth]{diffeq_plots/04_parameter_study.pdf}
    \caption{Vpliv parametrov: (a) različne amplitude $A$, (b) različni fazni zamiki $\delta$, (c) različni $k$, (d) razpon temperature v odvisnosti od parametrov}
    \label{fig:parameters}
\end{figure}

\newpage

\section{Analiza}

\subsection{Izbira metode in koraka}

Za večino aplikacij se je Runge-Kutta metoda 4. reda izkazala za najboljšo izbiro:
\begin{itemize}
    \item Zadostna natančnost že pri sorazmerno velikih korakih
    \item Stabilna tudi za daljše časovne intervale
    \item Učinkovita v računskem smislu
\end{itemize}

Pri zahtevah za visoko natančnost ($<10^{-6}$) je priporočljiv korak $h = 0.1$ ure.

\subsection{Periodično segrevanje}

Za analizo periodičnega segrevanja je pomembno:
\begin{itemize}
    \item Izbrati dovolj majhen korak za zajem hitrih sprememb
    \item Simulirati dovolj dolg čas za dosego stacionarnega stanja
    \item Uporabiti metodo, ki ohranja amplitudo nihanj
\end{itemize}

\subsection{Numerična stabilnost}

Eulerjeva metoda je pokazala največjo občutljivost na velikost koraka. 
Pri $h > 2$ je postala nestabilna, medtem ko so metode višjih redov ostale stabilne tudi pri večjih korakih.

\newpage

\section{Zaključek}

Preučili smo različne numerične metode za reševanje enačb hoda in dosegli naslednje ugotovitve:

\begin{itemize}
    \item \textbf{Runge-Kutta 4. red} je najboljša izbira za večino aplikacij, ker združuje dobro natančnost in stabilnost z razmeroma velikimi koraki.
    \item Za osnovno enačbo ohlajanja je zadosten korak $h = 0.5$ ure za natančnost $10^{-4}$.
    \item Pri periodičnem segrevanju je potreben manjši korak ($h = 0.1$ ure) za natančno zajemanje maksimumov.
    \item Vse metode so sposobne doseči absolutno napako pod $10^{-10}$ z dovolj majhnim korakom.
    \item Relativna napaka je bolj zahteven kriterij, vendar je tudi ta dosegljiv z ustreznimi metodami.
\end{itemize}

Numerične metode za reševanje diferencialnih enačb so močno orodje za modeliranje fizikalnih procesov. 
Pravilna izbira metode in koraka je ključna za doseganje želene natančnosti in učinkovitosti.

\end{document}
