\documentclass[12pt,a4paper]{article}
\usepackage[utf8]{inputenc}
\usepackage{amsmath, amssymb}
\usepackage{graphicx}
\usepackage{hyperref}
\usepackage{booktabs}
\usepackage{siunitx}
\usepackage{caption}
\usepackage{subcaption}
\usepackage{geometry}

\geometry{a4paper, margin=1.5cm}

\title{Enačbe hoda}
\author{Filip Jesenšek (28231064)}
\date{\today}

\begin{document}

\maketitle

\begin{abstract}
V tem poročilu podrobno analiziramo numerične metode za reševanje navadnih diferencialnih enačb 
s poudarkom na enačbah hoda za fizikalne modele ohlajanja. 
Preučujemo osnovno enačbo ohlajanja in njeno razširitev s periodičnim segrevanjem. 
Sistematično primerjamo Eulerjevo metodo, Midpoint metodo, Heunovo metodo in Runge-Kutta metodo 
4. reda. 
Analiziramo njihovo natančnost, stabilnost, konvergenco in računsko učinkovitost za 
različne vrednosti parametrov. 
Rezultati kažejo, da Runge-Kutta metoda 4. reda doseže najboljšo natančnost, 
medtem ko Eulerjeva metoda predstavlja najboljšo izbiro za hitre približke.
\end{abstract}

\tableofcontents

\newpage

\section{Uvod}

Enačbe hoda predstavljajo temeljni orodje za modeliranje časovnega razvoja fizikalnih sistemov. 
Te enačbe opisujejo, kako se stanje sistema spreminja v času glede na trenutno stanje in zunanje vplive. 
V fizikalnem kontekstu se pogosto uporabljajo za modeliranje procesov, kot so ohlajanje teles, 
rast populacij, kemijske reakcije in številni drugi dinamični sistemi.

\subsection{Teoretično ozadje}

Osnovna enačba ohlajanja, ki jo obravnavamo v tem delu, je podana z diferencialno enačbo prvega reda:

\begin{equation}
\frac{dT}{dt} = -k(T - T_{\mathrm{zun}})
\end{equation}

kjer $T$ predstavlja temperaturo sistema, $T_{\mathrm{zun}}$ zunanjo temperaturo okolice, 
$k$ pa parameter ohlajanja, ki določa hitrost prenosa toplote. 
Ta enačba ima analitično rešitev oblike:

\begin{equation}
T(t) = T_{\mathrm{zun}} + e^{-kt}(T(0) - T_{\mathrm{zun}})
\end{equation}

ki omogoča direktno primerjavo z numeričnimi metodami.
V nadaljevanju razširimo osnovni model z upoštevanjem periodičnega segrevanja, 
kar bolj ustreza realnim pogojem, kjer se temperatura okolice spreminja s časom. 
Razširjena enačba ima obliko:

\begin{equation}
\frac{dT}{dt} = -k(T - T_{\mathrm{zun}}) + A\sin\left(\frac{2\pi}{24}(t-\delta)\right)
\end{equation}

kjer $A$ predstavlja amplitudo segrevanja in $\delta$ fazni zamik. 
Ta model simulira dnevna nihanja temperature, kar je še posebej relevantno za aplikacije 
v klimatologiji in energetski učinkovitosti stavb.

\subsection{Numerične metode}

V študiji sistematično primerjamo naslednje numerične metode za reševanje diferencialnih enačb:
Eulerjeva metoda predstavlja najenostavnejši pristop s konvergenco prvega reda. 
Kljub nizki natančnosti je široko uporabljena zaradi enostavne implementacije 
in nizke računske zahtevnosti.
Midpoint metoda in Heunova metoda sta metodi drugega reda, ki izboljšata natančnost 
z uporabo dodatnih izračunov. 
Medtem ko Midpoint metoda uporablja vmesno točko za oceno naklona, 
Heunova metoda kombinira Eulerjevo napoved s popravkom.
Runge-Kutta metoda 4. reda predstavlja zlati standard za večino aplikacij, 
saj doseže visoko natančnost s konvergenco četrtega reda, 
ob ohranjanju razmeroma nizke računske zahtevnosti v primerjavi z višjimi redovi.

\section{Metodologija}

\subsection{Implementacija metod}

Za reševanje enačb smo implementirali vse štiri metode v programskem jeziku Python 
z uporabo knjižnic NumPy za numerične izračune in Matplotlib za vizualizacijo rezultatov. 
Uporabili smo naslednje parametre, ki so reprezentativni za realne scenarije ohlajanja:
Začetni temperaturi $T(0) = 21^\circ$C in $T(0) = -15^\circ$C 
pokrivata tipične scenarije segretega in ohlajenega začetnega stanja. 
Zunanja temperatura $T_{\mathrm{zun}} = -5^\circ$C ustreza zimskim razmeram, 
parameter ohlajanja $k = 0.1$ pa zagotavlja realistično hitrost ohlajanja. 
Časovni razpon $t \in [0, 50]$ ur omogoča opazovanje tako kratkoročnega odziva 
kot dolgoročnega ravnotežnega stanja.

\subsection{Analiza napak}

Za vsako metodo smo sistematično analizirali različne vidike napak in učinkovitosti. 
Absolutna napaka je definirana kot razlika med numerično in analitično rešitvijo, 
medtem ko relativna napaka predstavlja to razliko glede na velikost analitične rešitve. 
Odvisnost napake od velikosti koraka je ključnega pomena za razumevanje konvergence metod, 
saj nam omogoča določiti optimalno razmerje med natančnostjo in računsko zahtevnostjo.
Konvergenca numeričnih metod je analizirana s preučevanjem hitrosti, 
s katero se napaka zmanjšuje z zmanjševanjem velikosti koraka. 
Računska učinkovitost je ovrednotena z merjenjem časa izvajanja 
za različne velikosti korakov in število iteracij.

\newpage

\section{Rezultati}

\subsection{Primerjava metod za osnovno enačbo}

Na sliki \ref{fig:basic_comparison} lahko opazimo, kako različne numerične metode 
sledijo analitični rešitvi osnovne enačbe ohlajanja. 
Vse metode kvalitatívno pravilno reproducirajo eksponentno upadanje temperature 
proti zunanji temperaturi, vendar se razlikujejo v kvantitativni natančnosti.
Eulerjeva metoda kaže največje odstopanje od analitične rešitve, 
kar je pričakovano glede na njen nizki konvergenčni red. 
Midpoint in Heunova metoda izboljšata natančnost, 
vendar še vedno opazimo manjša odstopanja pri večjih korakih. 
Runge-Kutta metoda 4. reda doseže izjemno natančnost tudi pri razmeroma velikih korakih, 
kar potrjuje njeno superiornost za aplikacije, ki zahtevajo visoko natančnost.

\begin{figure}[htbp]
    \centering
    \includegraphics[width=0.9\textwidth]{diffeq_plots/01_basic_cooling_comparison.pdf}
    \caption{Primerjava numeričnih metod za osnovno enačbo ohlajanja. 
             Vse metode sledijo analitični rešitvi, 
             Runge-Kutta metoda 4. reda kaže najboljšo natančnost.}
    \label{fig:basic_comparison}
\end{figure}

\subsection{Analiza konvergence}

Analiza konvergence, prikazana na sliki \ref{fig:convergence_analysis}, 
razkriva temeljne lastnosti posameznih numeričnih metod. 
Zapis v logaritemski skali omogoča jasno razlikovanje konvergenčnih redov posameznih metod.
Eulerjeva metoda kaže linearno konvergenco, kar ustreza teoretičnemu redu $\mathcal{O}(h)$. 
Midpoint in Heunova metoda izkazujeta kvadratično konvergenco 
z eksperimentalnima redoma $\mathcal{O}(h^{1.85})$ oziroma $\mathcal{O}(h^{1.88})$, 
kar se dobro ujema s teoretičnimi pričakovanji. 
Runge-Kutta metoda 4. reda doseže najhitrejšo konvergenco 
z eksperimentalnim redom $\mathcal{O}(h^{3.65})$, 
kar je nekoliko nižje od teoretičnega $\mathcal{O}(h^4)$, 
vendar še vedno znatno presega ostale metode.

\begin{figure}[htbp]
    \centering
    \includegraphics[width=0.9\textwidth]{diffeq_plots/02_step_size_dependence.pdf}
    \caption{Analiza konvergence numeričnih metod. 
             Metode kažejo konvergenco, ki se približuje teoretičnim redom.}
    \label{fig:convergence_analysis}
\end{figure}

\begin{table}[htbp]
    \centering
    \begin{tabular}{lcc}
        \toprule
        Metoda & Teoretični red & Eksperimentalni red \\
        \midrule
        Euler & $\mathcal{O}(h)$ & $\mathcal{O}(h^{0.95})$ \\
        Midpoint & $\mathcal{O}(h^2)$ & $\mathcal{O}(h^{1.85})$ \\
        Heun & $\mathcal{O}(h^2)$ & $\mathcal{O}(h^{1.88})$ \\
        RK4 & $\mathcal{O}(h^4)$ & $\mathcal{O}(h^{3.65})$ \\
        \bottomrule
    \end{tabular}
    \caption{Primerjava teoretičnih in eksperimentalnih konvergenčnih redov}
    \label{tab:convergence_rates}
\end{table}

\subsection{Periodično segrevanje}

Za enačbo s periodičnim segrevanjem ($A=1$, $\delta=10$) 
opazimo zanimivo dinamiko, prikazano na sliki \ref{fig:periodic_heating}. 
Sistem sprva kaže prehodno obnašanje, kjer prevladuje vpliv začetnih pogojev, 
nato pa se postopoma vzpostavi periodično ustaljeno stanje.
V ustaljenem stanju sistem sledi dnevni periodi z nihanji temperature 
okoli povprečne vrednosti. 
Analiza kaže, da sistem doseže ustaljeno stanje po približno 3 dneh, 
kar ustreza karakterističnemu času $1/k = 10$ ur. 
Povprečna perioda nihanj znaša $24.0$ ur, kar potrjuje, 
da model pravilno zajema dnevna nihanja temperature. 
Amplituda nihanj v ustaljenem stanju je $2.1^\circ$C, 
kar odraža ravnovesje med segrevalnim členom in ohlajanjem.

\begin{figure}[htbp]
    \centering
    \includegraphics[width=0.9\textwidth]{diffeq_plots/03_periodic_heating.pdf}
    \caption{Rešitev enačbe s periodičnim segrevanjem. 
             Sistem doseže periodično ustaljeno stanje s karakteristično dnevno periodo.}
    \label{fig:periodic_heating}
\end{figure}

\subsection{Vpliv parametrov}

Študija vpliva parametrov, prikazana na sliki \ref{fig:parameter_study}, 
razkriva, kako posamezni parametri vplivajo na obnašanje sistema. 
Večja amplituda $A$ povzroči večja nihanja temperature, 
saj poveča intenziteto periodičnega segrevanja. 
Fazni zamik $\delta$ vpliva na časovni zamik maksimalnih temperatur, 
kar omogoča modeliranje različnih faz glede na sončno svetlobo.
Parameter $k$ ima ključno vlogo pri določanju hitrosti odziva sistema 
na zunanje vplive. 
Večji $k$ pomeni hitrejše prilagajanje temperature spremembam okolice, 
medtem ko manjši $k$ vodi v počasnejši odziv in večjo inerčnost sistema. 
To je še posebej pomembno pri modeliranju toplotne zmogljivosti stavb 
in drugih sistemov z veliko toplotno maso.

\begin{figure}[htbp]
    \centering
    \includegraphics[width=0.9\textwidth]{diffeq_plots/04_parameter_study.pdf}
    \caption{Vpliv parametrov na rešitev enačbe s periodičnim segrevanjem.}
    \label{fig:parameter_study}
\end{figure}

\subsection{Računska učinkovitost}

Analiza računske učinkovitosti, prikazana na sliki \ref{fig:efficiency} 
in tabeli \ref{tab:efficiency}, ponuja vpogled v praktične vidike uporabe 
različnih numeričnih metod. 
Eulerjeva metoda je najhitrejša, vendar zahteva manjše korake 
za dosego sprejemljive natančnosti. 
Midpoint in Heunova metoda ponujata dobro ravnovesje med hitrostjo in natančnostjo, 
kar jih naredi primerni za številne aplikacije.
Runge-Kutta metoda 4. reda, čeprah počasnejša od preostalih metod, 
doseže najboljšo natančnost na enoto računskega časa pri večjih korakih. 
To jo naredi posebej primerno za aplikacije, kjer je natančnost ključnega pomena 
in kjer lahko uporaba večjih korakov kompenzira povečano računsko zahtevnost na korak.

\begin{figure}[htbp]
    \centering
    \includegraphics[width=0.9\textwidth]{diffeq_plots/06_computational_efficiency.pdf}
    \caption{Analiza računske učinkovitosti. 
             Runge-Kutta metoda 4. reda doseže najboljšo ravnovesje 
             med natančnostjo in računsko zahtevnostjo.}
    \label{fig:efficiency}
\end{figure}

\begin{table}[htbp]
    \centering
    \begin{tabular}{lcc}
        \toprule
        Metoda & Hitrost & Natančnost \\
        \midrule
        Euler & Najhitrejša & Najmanjša \\
        Midpoint & Hitra & Dobra \\
        Heun & Srednja & Dobra \\
        RK4 & Počasnejša & Najboljša \\
        \bottomrule
    \end{tabular}
    \caption{Primerjava računske učinkovitosti metod}
    \label{tab:efficiency}
\end{table}

\newpage

\section{Razprava}

\subsection{Izbira metode}

Na podlagi rezultatov lahko podamo naslednja priporočila za izbiro numerične metode 
glede na specifične zahteve aplikacije:
Eulerjeva metoda je primerna za hitre približke in kvalitativno analizo, 
še posebej pri začetnem razvoju modelov, ko je hitrost pomembnejša od natančnosti. 
Njena enostavnost omogoča tudi lažje razumevanje in debugiranje kode.
Midpoint in Heunova metoda predstavljata odlično izbiro za aplikacije, 
ki zahtevajo dobro natančnost ob razmeroma nizki računski zahtevnosti. 
Te metode so še posebej primerne za sisteme z gladkimi rešitvami, 
kjer ni potrebe po ekstremni natančnosti.
Runge-Kutta metoda 4. reda je najboljša izbira za visoko natančnost 
in zahtevne aplikacije, kjer je natančnost ključnega pomena. 
Njena robustnost in zanesljivost jo naredita primerno za produkcijske sisteme 
in znanstvene simulacije, kjer so natančni rezultati esencialni.

\subsection{Vpliv koraka}

Velikost koraka ima ključno vlogo pri ravnovesju med natančnostjo in računsko zahtevnostjo. 
Za osnovno enačbo ohlajanja je korak $h = 0.5$ ure zadosten 
za doseganje natančnosti reda $10^{-4}$ z Runge-Kutta metodo 4. reda. 
Za periodično segrevanje pa je potreben manjši korak $h = 0.1$ ure 
za natančno zajemanje maksimumov in dinamike nihanj.
Pri izbiri koraka je pomembno upoštevati tudi stabilnost metode. 
Eulerjeva metoda lahko postane nestabilna pri prevelikih korakih, 
medtem ko so višji redovi metod bolj robustni v tem pogledu. 
Praksa kaže, da je koristno izvesti konvergenčno analizo za vsak nov problem, 
da se določi optimalna velikost koraka.

\section{Zaključek}

Preučili smo različne numerične metode za reševanje enačb hoda 
in dosegli naslednje glavne ugotovitve:
Runge-Kutta metoda 4. reda je najboljša izbira za večino aplikacij, 
ki zahtevajo visoko natančnost. 
Njena superiornost se kaže še posebej pri večjih korakih, 
kjer ohranja izjemno natančnost ob sorazmerno nizki dodatni računski zahtevnosti.
Vse metode so sposobne doseči absolutno napako pod $10^{-10}$ 
z dovolj majhnim korakom, kar potrjuje univerzalnost numeričnih pristopov 
za reševanje diferencialnih enačb. 
Vendar se metode bistveno razlikujejo v hitrosti konvergence in računski učinkovitosti.
Za periodične procese je potreben manjši korak za natančno zajemanje dinamike, 
kar poudarja pomembnost prilagajanja numeričnih parametrov specifični naravi problema. 
Izbira metode je odvisna od zahtev glede natančnosti in računske zahtevnosti, 
pri čemer je ključno najti ravnovesje med tehničnimi zahtevami 
in razpoložljivimi računskimi viri.
Za nadaljnje delo bi bilo zanimivo razširiti analizo na sisteme diferencialnih enačb 
in preučiti obnašanje metod pri problemih z več prostostnimi stopnjami. 
Prav tako bi bilo koristno preučiti adaptivne metode, 
ki samodejno prilagajajo velikost koraka glede na lokalno napako, 
kar bi omogočilo še boljšo učinkovitost pri kompleksnih problemih.

\end{document}
